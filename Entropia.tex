\documentclass[12pt]{article}
\usepackage{amsmath}
\usepackage{multicol}
\usepackage[latin1]{inputenc}
\usepackage[brazil]{babel}
\usepackage{graphicx}


\begin{document}

% CAPA
\begin{titlepage} 
\begin{center} 
{\large FATEC Rubens Lara}\\[0.2cm] 
{\large Ci\^{e}ncia de Daods}\\[0.2cm]
{\large Matem\'{a}tica B\'{a}sica}\\[4.1cm]
{\bf \huge Entropia de Dados}\\[4.1cm] 
{\large Enri Lopes Iwasaki}\\[0.5cm] 
{\large Leandro Costa Santos}\\[4.7cm]
{\large Santos}\\[0.2cm]
{\large 2023}
\end{center}
\end{titlepage}

\newpage

    A base de dados utilizada foi obtida por meio do {\bfseries IGBE - Instituto Brasileiro de Geografia e Estat\'{i}stica}. A base apresenta dados referentes \‘{a} {\bfseries S\'{i}ntese de Indicadores Sociais}, que analisa a qualidade de vida e os n\'{i}veis de bem-estar das pessoas, fam\'{i}lias e grupos populacionais, a efetiva\c{c}\~{a}o de direitos humanos e sociais, bem como o acesso a diferentes servi\c{c}os, bens e oportunidades, por meio de indicadores que visam contemplar a heterogeneidade da sociedade brasileira sob a perspectiva das desigualdades sociais.
\paragraph{}
    Os dados verificados abordam, mais precisamente, a {\bfseries Total e respectiva distribui\c{c}\~{a}o percentual das pessoas, por classes de rendimento no ano de 2022}. Estando dispon\'{i}vel em: https://www.ibge.gov.br/estatisticas/
    sociais/trabalho/9221sintese\_ de\_indicadores\_sociais.html?=\&t=resultados.

\begin{figure} [h]
\\
    \includegraphics[width=14cm, height=2cm]{IBGE.png}
\end{figure}

\begin{center}
&\begin{array}{|c|c|c|}
    \hline \multicolumn{3}{|c|}{\text { Popula\c{c}\~{a}o Total do Brasil }} \\
    \hline \multicolumn{3}{|c|}{212.577 .000} \\
    \hline \text { Rendimento em Sal\'{a}rios Recebidos } & \text { Popula\c{c}\~{a}o } & \text { Porcentagem }(\%) \\
    \hline 0 & 4.251 .540 & 2 \\
    \hline 0-1 & 130.522 .278 & 61,4 \\
    \hline 1-2 & 48.042 .402 & 22,6 \\
    \hline 2-3 & 13.817 .505 & 6,5 \\
    \hline 3-5 & 8.928 .234 & 4,2 \\
    \hline>5 & 7.015 .041 & 3,3 \\
    \hline
\end{array}\\
\end{center}


\begin{center} 
$$
    & Classes=\left\{{ }'0' \ , '0-1' \ , '1-2' \ , '2-3' \ , '3-5' \ , '>5' }\right\} \\ 
    \qquad \qquad { \mid Classes \mid }=6
$$
\end{center}

\newpage

\begin{center}
\[
    \mathrm{H}(X)=-\sum_{x \ \in \ classes} \mathrm{P}\left(x\right) \log _2 \mathrm
    \left(P \left(x\right)\right) \\
\]
\end{center}

\begin{center}
$
\\
   \quad \mathrm{H}=-\left(0,02 \log _2^{0,02}+0,614 \log _2^{0,614}+0,226
    \log _2^{0,226}+\right\\ \quad \qquad 0,65 \log _2^{0,065}+0,042 \log _2^{0,042}+0,033 \log _2^{0,033)} 
$
\end{center}

\begin{aligned}
    \quad \log _2^{0,02}=\frac{\log 0,02}{\log 2}=-5,64 \qquad \log _2 0,614=\frac{\log 0,614}{\log 2}=-0,70 \\
    \\
    \log _2^{0,226}=\frac{\log 0,226}{\log 2}=-2,15 \qquad \log _2^{0,065}=\frac{\log 0,065}{\log 2}=-3,94\\
    \\
    \log _2^{0,042}=\frac{\log 0,042}{\log 2}=-4,57 \qquad \log _2^{0,033}=\frac{\log 0,033}{\log 2}=-4,92 \\
    \\
\end{aligned}

\begin{aligned}
\[
    & \quad \mathrm{H}=-(0,02 \cdot(-5,64)+0,614 \cdot(-0,70)+0,0226 \cdot(-2,15)+ \\
    & \quad \qquad 0,065 \cdot(-3,94)+0,042 \cdot(-4,57)+0,033 \cdot(-4,92)) \\
\]

\begin{center}
    \fbox{ $$ \mathrm{H}=1,64 $$ } \\
\end{center}

\begin{center}
$$$$
    {\bfseries Entropia M\'{a}xima dos Dados}
\[
    \operatorname{Max}_{\mathrm{H}} = \log _2 ^ {\mid classes \mid}=\log _2 ^ 6=\frac{\log 6}{\log 2}
\]
    \fbox{ $$ \ \operatorname{Max}{_\mathrm{H}}=2,58 \quad $$ } 
\end{center}

\newpage

Programa\c{c}\~{a}o do m\'{e}todo em Python:
\\

\begin{figure} [h]
    \includegraphics[width=15cm, height=13cm]{determinantes.png}
\end{figure}

\newpage
Demonstra\c{c}\~{a}o do Console:
\begin{figure} [h]
    \includegraphics[width=7.5cm, height=14cm]{DET 1.jpg}
\end{figure}

\end{document}


